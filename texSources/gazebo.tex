\section{Gazebo}
La parte correspondiente al control del robot en el entorno de Gazebo consiste en dos partes, por un lado se leen los diferentes \textit{topics} que envía todo el sistema de visión y por otro se interactúa con el robot en función a estos valores.

\subsection{Abrir/cerrar pinza}
Según el valor obtenido por el \textit{topic} relacionado a esta acción (sección \ref{sec:unionRos}) se manda una señal al \textit{topic} del robot \textit{/parallel\_gripper\_controller/command}  publicando dos valores diferentes según si se ha de abrir o cerrar.

\subsection{Mover base}
En función del valor entero recibido del topic correspondiente se codifica según se menciona en el apartado (sección \ref{sec:unionRos}) y se publica en el \textit{topic} del robot \textit{/mobile\_base\_controller/cmd\_vel} uno de los siguientes valores:
\begin{itemize}
  \item Velocidades angulares y lineales = 0
  \item Velocidad angular = 0 y lineal = 1
  \item Velocidad angular = 1 y lineal = 0
\end{enumerate}
\subsection{Mover brazo}
En este caso se parte del ejemplo de MoveIt que define el movimiento cartesiano de un robot pasándole como argumentos los valores (x, y, z) y (roll, pitch, yaw) pero realizando ciertas modificaciones.\\

En lugar de leer los valores de (x, y, z) como argumentos los extrae de los topics \textit{/cartPosition/x}, \textit{/cartPosition/y} y /cartPosition/z respectivamente, en cuanto a los valores de roll, pitch y yaw se le da valores por defectos tales como 1.57, 0 y 0 respectivamente.\\

Con estos valores de posición y orientación se crea un planificador de trayectorias cartesianas haciendo uso de un solver de cinemática inversa. Dicho solvert planea la trayectoria desde su posición actual a la posición destino dada y, pasado un pequeño intervalo de tiempo y en caso de que no encuentre ninguna singularidad a lo largo de la trayectoria, realiza el movimiento solicitado.\\

