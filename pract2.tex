\documentclass[12pt, a4paper]{article}
\usepackage[utf8]{inputenc}
\usepackage[spanish]{babel}
\usepackage{graphicx}
\usepackage{subfigure}
\usepackage{hyperref}
\usepackage[center]{caption}
\usepackage{xcolor}

\title{\huge Sistemas de Percepción\\ \LARGE Estado proyecto.}
\author{Alfredo Fenoll Quirant \\Francisco Morillas Espejo \\ Raúl Candela Arias}
\date{}
 
\begin{titlepage}
\maketitle
\end{titlepage}
 
\begin{document}

Se ha comenzado el proyecto por la parte de simulación de un robot humanoide (Pepper o TiaGo) mediante Gazebo.
\newline

A lo largo de esta parte se han encontrado diversos errores debido a la versión de Gazebo, funcionando unas partes de los robots bien en una u otra versión pero sin conseguir que funcionen al completo.
Por tanto una de los principales problemas que se está tratando de solventar es la simulación de dichos robots.
\newline

Otra de las opciones que se ha barajado es el uso de un robot simple, mediante eslabones cilíndricos, para reducir la complejidad del problema.
\newline

Tras esto se plantea el uso de OpenPose para la detección del movimiento de los brazos.
Cada movimiento será posterioremente enviado al robot simulado el cual realizará la acción correspondiente (moverse, mover brazo, abrir/cerrar efector final).
Con todo esto realizado el siguiente paso es dejar OpenPose y realizar la detección de movimientos mediante las técnicas de visión vistas en la asignatura.
 
\end{document}
